\usepackage{float}
\usepackage[]{graphicx}
\graphicspath{ {fig/} }
\usepackage{enumitem}
\usepackage{xcolor}
\usepackage[colorinlistoftodos,prependcaption]{todonotes}


\newcommand{\dat}[1]{\todo[color=green!40]{#1}}
\newcommand{\simon}[1]{\todo[color=blue!40]{#1}}

\startlocaldefs
\numberwithin{equation}{section}
\theoremstyle{plain}
\newtheorem{thm}{Theorem}[section]
\newtheorem{lem}{Lemma}[section]
\newtheorem{cor}{Corollary}[section]
\newtheorem{prop}{Proposition}[section]
\theoremstyle{remark}
\newtheorem{rem}{Remark}[section]

% -----------------------------------------------------------------------------
% algorithms
\usepackage[]{algorithmicx}
\usepackage[]{algorithm}
\usepackage{algpseudocode}
\algnewcommand\algorithmicparameters{\textbf{Parameters:}}
\algnewcommand\PARAMETERS{\item[\algorithmicparameters]}
\renewcommand\algorithmicprocedure{\textbf{Procedure:}}
\algnewcommand\PROCEDURE{\item[\algorithmicprocedure]}
\algnewcommand\algorithmicendprocedure{\textbf{End Procedure}}
\algnewcommand\ENDPROCEDURE{\item[\algorithmicendprocedure]}
\algnewcommand\algorithmicinput{\textbf{Input:}}
\algnewcommand\INITIALIZE{\item[\algorithmicinitialize]}
\algnewcommand\algorithmicinitialize{\textbf{Initialize:}}
\algnewcommand\INPUT{\item[\algorithmicinput]}
\algnewcommand\algorithmicoutput{\textbf{Output:}}
\algnewcommand\OUTPUT{\item[\algorithmicoutput]}
\renewcommand{\algorithmicwhile}{\textbf{While}}
\renewcommand{\algorithmicend}{\textbf{End}}
\renewcommand{\algorithmicif}{\textbf{If}}
\renewcommand{\algorithmicelse}{\textbf{Else}}
\renewcommand{\algorithmicrepeat}{\textbf{Repeat:}}
\renewcommand{\algorithmicuntil}{\textbf{Until}}
\renewcommand{\algorithmicfor}{\textbf{For}}
\renewcommand{\algorithmicdo}{\textbf{do:}}


% -----------------------------------------------------------------------------
% math macros
\usepackage{amsfonts}
\usepackage{amssymb}
\usepackage{amsthm}
\usepackage{bbm}
\usepackage{amsmath}
% general
\DeclareMathOperator{\eps}{\varepsilon}
% dx with spacing
\makeatletter \renewcommand\d[1]{\ensuremath{%
  \,\mathrm{d}#1\@ifnextchar\d{\!}{}}}
\makeatother
% norms
\newcommand{\vertiii}[1]{{\left\vert\kern-0.25ex\left\vert\kern-0.25ex\left\vert #1 
    \right\vert\kern-0.25ex\right\vert\kern-0.25ex\right\vert}}
\newcommand{\vertii}[1]{{\left\vert\kern-0.25ex\left\vert #1 
    \right\vert\kern-0.25ex\right\vert}}
\newcommand{\verti}[1]{{\left\vert #1 
    \right\vert}}
% inner product
\newcommand{\inprod}[1]{{\left\langle #1 
    \right\rangle}}
% text operators
\DeclareMathOperator{\ind}{\mathbbm 1}
\DeclareMathOperator{\ddd}{,\ldots ,}
\DeclareMathOperator{\TV}{TV}
\DeclareMathOperator{\dist}{dist}
\DeclareMathOperator{\diag}{diag}
\DeclareMathOperator{\im}{Im}
\DeclareMathOperator{\iid}{i.i.d.}
\DeclareMathOperator{\MH}{MH}
\DeclareMathOperator{\tr}{tr}
\DeclareMathOperator{\MTM}{MTM}
\DeclareMathOperator{\prox}{\bf prox}
%caligraphic fonts
\DeclareMathOperator{\cA}{\mathcal A}
\DeclareMathOperator{\cB}{\mathcal B}
\DeclareMathOperator{\cC}{\mathcal C}
\DeclareMathOperator{\cD}{\mathcal D}
\DeclareMathOperator{\cF}{\mathcal F}
\DeclareMathOperator{\cG}{\mathcal G}
\DeclareMathOperator{\cH}{\mathcal H}
\DeclareMathOperator{\cI}{\mathcal I}
\DeclareMathOperator{\cK}{\mathcal K}
\DeclareMathOperator{\cL}{\mathcal L}
\DeclareMathOperator{\cN}{\mathcal N}
\DeclareMathOperator{\cO}{\mathcal O}
\DeclareMathOperator{\cP}{\mathcal P}
\DeclareMathOperator{\cQ}{\mathcal Q}
\DeclareMathOperator{\cS}{\mathcal S}
\DeclareMathOperator{\cT}{\mathcal T}
\DeclareMathOperator{\cW}{\mathcal W}
\DeclareMathOperator{\cX}{\mathcal X}
\DeclareMathOperator{\cY}{\mathcal Y}
%bold symbols
\DeclareMathOperator{\b0}{\mathbf 0}
\DeclareMathOperator{\b1}{\mathbf 1}
\DeclareMathOperator{\bX}{\mathbf X}
\DeclareMathOperator{\bx}{\mathbf x}
\DeclareMathOperator{\bB}{\mathbf B}
\DeclareMathOperator{\bD}{\mathbf D}
\DeclareMathOperator{\bR}{\mathbf R}
\DeclareMathOperator{\bbeta}{\boldsymbol \beta}
%tilde and bold symbols
\DeclareMathOperator{\tD}{\tilde{\bD} }
%convergences
\DeclareMathOperator{\convP}{\overset{\bP}{\rightarrow}}
\DeclareMathOperator{\convps}{\overset{p.s.}{\rightarrow}}
\DeclareMathOperator{\convL}{\overset{\mathcal L}{\rightarrow}}
\DeclareMathOperator{\convL2}{\overset{\mathcal L_2}{\rightarrow}}
\DeclareMathOperator{\convD}{\overset{\mathcal D}{\rightarrow}}
%long convergence
\DeclareMathOperator{\lconvP}{\;\overset{\bP}{\longrightarrow}\;}
\DeclareMathOperator{\lconvps}{\;\overset{p.s.}{\longrightarrow}\;}
\DeclareMathOperator{\lconvL}{\;\overset{\mathcal L}{\longrightarrow}\;}
\DeclareMathOperator{\lconvL2}{\;\overset{\mathcal L_2}{\longrightarrow}\;}
\DeclareMathOperator{\lconvD}{\;\overset{\mathcal D}{\longrightarrow}\;}
%operators with indeces underneath
\DeclareMathOperator*{\argmax}{arg\,max}
\DeclareMathOperator*{\argmin}{arg\,min}
%blackbord fonts
\DeclareMathOperator{\bbE}{\mathbb E}
\DeclareMathOperator{\bbN}{\mathbb N}
\DeclareMathOperator{\bbR}{\mathbb R}
\DeclareMathOperator{\bbP}{\mathbb P}
\DeclareMathOperator{\bbZ}{\mathbb Z}
% -----------------------------------------------------------------------------